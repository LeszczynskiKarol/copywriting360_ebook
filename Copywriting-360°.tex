% Options for packages loaded elsewhere
\PassOptionsToPackage{unicode}{hyperref}
\PassOptionsToPackage{hyphens}{url}
%
\documentclass[
  11pt,
  a4paper,
  twoside,
  openright,
  11pt,
  DIV=12,
  BCOR=12mm]{scrbook}

\usepackage{amsmath,amssymb}
\usepackage{setspace}
\usepackage{iftex}
\ifPDFTeX
  \usepackage[T1]{fontenc}
  \usepackage[utf8]{inputenc}
  \usepackage{textcomp} % provide euro and other symbols
\else % if luatex or xetex
  \usepackage{unicode-math}
  \defaultfontfeatures{Scale=MatchLowercase}
  \defaultfontfeatures[\rmfamily]{Ligatures=TeX,Scale=1}
\fi
\usepackage{lmodern}
\ifPDFTeX\else  
    % xetex/luatex font selection
    \setmainfont[]{Libertinus Serif}
    \setsansfont[]{Fira Sans}
    \setmonofont[]{Fira Code}
\fi
% Use upquote if available, for straight quotes in verbatim environments
\IfFileExists{upquote.sty}{\usepackage{upquote}}{}
\IfFileExists{microtype.sty}{% use microtype if available
  \usepackage[]{microtype}
  \UseMicrotypeSet[protrusion]{basicmath} % disable protrusion for tt fonts
}{}
\makeatletter
\@ifundefined{KOMAClassName}{% if non-KOMA class
  \IfFileExists{parskip.sty}{%
    \usepackage{parskip}
  }{% else
    \setlength{\parindent}{0pt}
    \setlength{\parskip}{6pt plus 2pt minus 1pt}}
}{% if KOMA class
  \KOMAoptions{parskip=half}}
\makeatother
\usepackage{xcolor}
\usepackage[top=25mm,bottom=25mm,inner=25mm,outer=20mm]{geometry}
\setlength{\emergencystretch}{3em} % prevent overfull lines
\setcounter{secnumdepth}{3}
% Make \paragraph and \subparagraph free-standing
\makeatletter
\ifx\paragraph\undefined\else
  \let\oldparagraph\paragraph
  \renewcommand{\paragraph}{
    \@ifstar
      \xxxParagraphStar
      \xxxParagraphNoStar
  }
  \newcommand{\xxxParagraphStar}[1]{\oldparagraph*{#1}\mbox{}}
  \newcommand{\xxxParagraphNoStar}[1]{\oldparagraph{#1}\mbox{}}
\fi
\ifx\subparagraph\undefined\else
  \let\oldsubparagraph\subparagraph
  \renewcommand{\subparagraph}{
    \@ifstar
      \xxxSubParagraphStar
      \xxxSubParagraphNoStar
  }
  \newcommand{\xxxSubParagraphStar}[1]{\oldsubparagraph*{#1}\mbox{}}
  \newcommand{\xxxSubParagraphNoStar}[1]{\oldsubparagraph{#1}\mbox{}}
\fi
\makeatother


\providecommand{\tightlist}{%
  \setlength{\itemsep}{0pt}\setlength{\parskip}{0pt}}\usepackage{longtable,booktabs,array}
\usepackage{calc} % for calculating minipage widths
% Correct order of tables after \paragraph or \subparagraph
\usepackage{etoolbox}
\makeatletter
\patchcmd\longtable{\par}{\if@noskipsec\mbox{}\fi\par}{}{}
\makeatother
% Allow footnotes in longtable head/foot
\IfFileExists{footnotehyper.sty}{\usepackage{footnotehyper}}{\usepackage{footnote}}
\makesavenoteenv{longtable}
\usepackage{graphicx}
\makeatletter
\newsavebox\pandoc@box
\newcommand*\pandocbounded[1]{% scales image to fit in text height/width
  \sbox\pandoc@box{#1}%
  \Gscale@div\@tempa{\textheight}{\dimexpr\ht\pandoc@box+\dp\pandoc@box\relax}%
  \Gscale@div\@tempb{\linewidth}{\wd\pandoc@box}%
  \ifdim\@tempb\p@<\@tempa\p@\let\@tempa\@tempb\fi% select the smaller of both
  \ifdim\@tempa\p@<\p@\scalebox{\@tempa}{\usebox\pandoc@box}%
  \else\usebox{\pandoc@box}%
  \fi%
}
% Set default figure placement to htbp
\def\fps@figure{htbp}
\makeatother

% ============================================================
% COPYWRITING 360° - Professional Header
% ============================================================

% ------------------------------------------------------------
% COLORS
% ------------------------------------------------------------
\usepackage{xcolor}

\definecolor{primaryblue}{HTML}{1a73e8}
\definecolor{primarydark}{HTML}{0d47a1}
\definecolor{primarylight}{HTML}{e3f2fd}
\definecolor{accentgreen}{HTML}{00c853}
\definecolor{accentred}{HTML}{d50000}
\definecolor{accentpurple}{HTML}{7c4dff}
\definecolor{accentorange}{HTML}{ff6d00}
\definecolor{textdark}{HTML}{212121}
\definecolor{textmuted}{HTML}{757575}
\definecolor{backgroundgray}{HTML}{f5f5f5}
\definecolor{bordergray}{HTML}{e0e0e0}
\definecolor{quotebg}{HTML}{fff8e1}
\definecolor{quoteborder}{HTML}{ffab00}
\definecolor{tipbg}{HTML}{e8f5e9}
\definecolor{tipborder}{HTML}{4caf50}
\definecolor{warningbg}{HTML}{fff3e0}
\definecolor{warningborder}{HTML}{ff9800}
\definecolor{importantbg}{HTML}{fce4ec}
\definecolor{importantborder}{HTML}{e91e63}
\definecolor{exercisebg}{HTML}{e3f2fd}
\definecolor{exerciseborder}{HTML}{1976d2}

% ------------------------------------------------------------
% PACKAGES
% ------------------------------------------------------------
\usepackage{tcolorbox}
\tcbuselibrary{skins,breakable}
\usepackage{fontawesome5}
\usepackage{booktabs}
\usepackage{tabularx}
\usepackage{float}
\usepackage{fancyhdr}
\usepackage{titlesec}
\usepackage{enumitem}
\usepackage{tikz}
\usetikzlibrary{positioning}

% ------------------------------------------------------------
% CHAPTER & SECTION STYLING
% ------------------------------------------------------------
\titleformat{\chapter}[display]
  {\normalfont\huge\bfseries\color{primarydark}}
  {\flushright\fontsize{72}{72}\selectfont\textcolor{primaryblue!30}{\thechapter}}
  {-20pt}
  {\Huge}

\titleformat{\section}
  {\normalfont\Large\bfseries\color{primarydark}}
  {\textcolor{primaryblue}{\thesection}}
  {1em}
  {}

\titleformat{\subsection}
  {\normalfont\large\bfseries\color{textdark}}
  {\textcolor{primaryblue}{\thesubsection}}
  {0.75em}
  {}

\titlespacing*{\chapter}{0pt}{30pt}{20pt}

% ------------------------------------------------------------
% HEADERS & FOOTERS
% ------------------------------------------------------------
\pagestyle{fancy}
\fancyhf{}
\fancyhead[LE,RO]{\small\color{textmuted}\leftmark}
\fancyfoot[LE,RO]{\bfseries\color{primaryblue}\thepage}
\fancyfoot[RE,LO]{\small\color{textmuted}Copywriting 360°}
\renewcommand{\headrulewidth}{0.4pt}
\renewcommand{\footrulewidth}{0pt}

\fancypagestyle{plain}{
  \fancyhf{}
  \fancyfoot[C]{\bfseries\color{primaryblue}\thepage}
  \renewcommand{\headrulewidth}{0pt}
}

% ------------------------------------------------------------
% CUSTOM BOXES
% ------------------------------------------------------------

% Quote box
\newtcolorbox{cytat}{
  enhanced,
  breakable,
  colback=quotebg,
  colframe=quoteborder,
  boxrule=0pt,
  borderline west={4pt}{0pt}{quoteborder},
  sharp corners,
  left=15pt, right=15pt, top=12pt, bottom=12pt,
  fontupper=\itshape\large
}

% Tip box
\newtcolorbox{wskazowka}{
  enhanced,
  breakable,
  colback=tipbg,
  colframe=tipborder,
  boxrule=0pt,
  borderline west={4pt}{0pt}{tipborder},
  sharp corners,
  left=15pt, right=15pt, top=12pt, bottom=12pt,
  title={\color{tipborder}\faLightbulb[regular]\ \ WSKAZÓWKA},
  fonttitle=\bfseries,
  coltitle=tipborder,
  attach boxed title to top left={yshift=-2mm, xshift=5mm},
  boxed title style={colback=tipbg, boxrule=0pt}
}

% Warning box
\newtcolorbox{uwaga}{
  enhanced,
  breakable,
  colback=warningbg,
  colframe=warningborder,
  boxrule=0pt,
  borderline west={4pt}{0pt}{warningborder},
  sharp corners,
  left=15pt, right=15pt, top=12pt, bottom=12pt,
  title={\color{warningborder}\faExclamationTriangle\ \ UWAGA},
  fonttitle=\bfseries,
  coltitle=warningborder,
  attach boxed title to top left={yshift=-2mm, xshift=5mm},
  boxed title style={colback=warningbg, boxrule=0pt}
}

% Important box
\newtcolorbox{wazne}{
  enhanced,
  breakable,
  colback=importantbg,
  colframe=importantborder,
  boxrule=0pt,
  borderline west={4pt}{0pt}{importantborder},
  sharp corners,
  left=15pt, right=15pt, top=12pt, bottom=12pt,
  title={\color{importantborder}\faExclamationCircle\ \ WAŻNE},
  fonttitle=\bfseries,
  coltitle=importantborder,
  attach boxed title to top left={yshift=-2mm, xshift=5mm},
  boxed title style={colback=importantbg, boxrule=0pt}
}

% Exercise box
\newtcolorbox{cwiczenie}[1]{
  enhanced,
  breakable,
  colback=exercisebg,
  colframe=exerciseborder,
  boxrule=1pt,
  arc=5pt,
  left=15pt, right=15pt, top=12pt, bottom=12pt,
  title={\color{white}\faEdit\ \ ĆWICZENIE: #1},
  fonttitle=\bfseries,
  coltitle=white,
  attach boxed title to top left={yshift=-3mm, xshift=5mm},
  boxed title style={colback=exerciseborder, arc=3pt}
}

% Concept box
\newtcolorbox{pojecie}[1]{
  enhanced,
  breakable,
  colback=primarylight,
  colframe=primarydark,
  boxrule=2pt,
  arc=0pt,
  left=15pt, right=15pt, top=12pt, bottom=12pt,
  title={\color{white}\faKey\ \ #1},
  fonttitle=\bfseries,
  coltitle=white,
  attach boxed title to top left={yshift=-3mm, xshift=5mm},
  boxed title style={colback=primarydark, arc=2pt}
}

% Formula box
\newtcolorbox{formula}[1]{
  enhanced,
  colback=white,
  colframe=accentpurple,
  boxrule=2pt,
  arc=8pt,
  left=15pt, right=15pt, top=12pt, bottom=12pt,
  title={\color{white}\faMagic\ \ FORMUŁA: #1},
  fonttitle=\bfseries,
  coltitle=white,
  attach boxed title to top center={yshift=-3mm},
  boxed title style={colback=accentpurple, arc=5pt}
}

% Example box
\newtcolorbox{przyklad}[1]{
  enhanced,
  breakable,
  colback=backgroundgray,
  colframe=primaryblue,
  boxrule=1pt,
  arc=0pt,
  left=15pt, right=15pt, top=12pt, bottom=12pt,
  title={\color{white}\faFileAlt\ \ PRZYKŁAD: #1},
  fonttitle=\bfseries,
  coltitle=white,
  attach boxed title to top left={yshift=-3mm, xshift=5mm},
  boxed title style={colback=primaryblue, arc=2pt}
}

% ------------------------------------------------------------
% LIST STYLING
% ------------------------------------------------------------
\setlist[itemize,1]{label=\textcolor{primaryblue}{\faChevronRight}, leftmargin=*, itemsep=4pt}
\setlist[itemize,2]{label=\textcolor{primaryblue}{\faAngleRight}, leftmargin=*, itemsep=2pt}
\setlist[enumerate,1]{label=\textcolor{primaryblue}{\bfseries\arabic*.}, leftmargin=*, itemsep=4pt}

\newlist{checklist}{itemize}{1}
\setlist[checklist]{label=\textcolor{accentgreen}{\faCheckSquare}, leftmargin=*, itemsep=4pt}

% ------------------------------------------------------------
% SPECIAL COMMANDS
% ------------------------------------------------------------

% Chapter opener quote
\newcommand{\chapteropener}[1]{%
  \begin{center}
    \large\itshape\color{textmuted}#1
  \end{center}
  \vspace{1em}
}

% Decorative line
\newcommand{\decoline}{%
  \begin{center}
    \textcolor{bordergray}{\rule{0.25\textwidth}{0.5pt}}%
    \quad\textcolor{primaryblue}{\faFeather}\quad%
    \textcolor{bordergray}{\rule{0.25\textwidth}{0.5pt}}
  \end{center}
}

% Key takeaway
\newcommand{\takeaway}[1]{%
  \begin{tcolorbox}[
    enhanced,
    colback=accentgreen!10,
    colframe=accentgreen,
    boxrule=0pt,
    borderline south={3pt}{0pt}{accentgreen},
    sharp corners,
    left=15pt, right=15pt, top=10pt, bottom=10pt
  ]
    \textbf{\color{accentgreen}\faCheckCircle\ \ KLUCZOWY WNIOSEK:}\\[5pt]
    #1
  \end{tcolorbox}
}

% Stat highlight
\newcommand{\stathighlight}[2]{%
  \begin{center}
    \begin{tcolorbox}[
      enhanced,
      colback=primarylight,
      colframe=primaryblue,
      boxrule=1pt,
      arc=5pt,
      width=0.6\textwidth,
      halign=center
    ]
      {\fontsize{36}{40}\selectfont\bfseries\color{primarydark}#1}\\[5pt]
      {\color{textmuted}#2}
    \end{tcolorbox}
  \end{center}
}

% Before/After comparison
\newcommand{\beforeafter}[2]{%
  \begin{tcolorbox}[
    enhanced,
    sidebyside,
    sidebyside align=top,
    lefthand width=0.47\textwidth,
    colback=white,
    colframe=bordergray,
    boxrule=0.5pt,
    arc=5pt,
    left=10pt, right=10pt, top=10pt, bottom=10pt
  ]
    \textbf{\color{accentred}\faTimesCircle\ \ PRZED}\\[6pt]
    \small #1
    \tcblower
    \textbf{\color{accentgreen}\faCheckCircle\ \ PO}\\[6pt]
    \small #2
  \end{tcolorbox}
}

% ------------------------------------------------------------
% TABLE STYLING
% ------------------------------------------------------------
\renewcommand{\arraystretch}{1.3}
\makeatletter
\@ifpackageloaded{bookmark}{}{\usepackage{bookmark}}
\makeatother
\makeatletter
\@ifpackageloaded{caption}{}{\usepackage{caption}}
\AtBeginDocument{%
\ifdefined\contentsname
  \renewcommand*\contentsname{Spis treści}
\else
  \newcommand\contentsname{Spis treści}
\fi
\ifdefined\listfigurename
  \renewcommand*\listfigurename{Spis rycin}
\else
  \newcommand\listfigurename{Spis rycin}
\fi
\ifdefined\listtablename
  \renewcommand*\listtablename{Spis tabel}
\else
  \newcommand\listtablename{Spis tabel}
\fi
\ifdefined\figurename
  \renewcommand*\figurename{Rysunek}
\else
  \newcommand\figurename{Rysunek}
\fi
\ifdefined\tablename
  \renewcommand*\tablename{Tabela}
\else
  \newcommand\tablename{Tabela}
\fi
}
\@ifpackageloaded{float}{}{\usepackage{float}}
\floatstyle{ruled}
\@ifundefined{c@chapter}{\newfloat{codelisting}{h}{lop}}{\newfloat{codelisting}{h}{lop}[chapter]}
\floatname{codelisting}{Wykaz}
\newcommand*\listoflistings{\listof{codelisting}{Spis wykazów}}
\makeatother
\makeatletter
\makeatother
\makeatletter
\@ifpackageloaded{caption}{}{\usepackage{caption}}
\@ifpackageloaded{subcaption}{}{\usepackage{subcaption}}
\makeatother

\ifLuaTeX
\usepackage[bidi=basic]{babel}
\else
\usepackage[bidi=default]{babel}
\fi
\babelprovide[main,import]{polish}
\ifPDFTeX
\else
\babelfont{rm}[]{Libertinus Serif}
\fi
% get rid of language-specific shorthands (see #6817):
\let\LanguageShortHands\languageshorthands
\def\languageshorthands#1{}
\usepackage{bookmark}

\IfFileExists{xurl.sty}{\usepackage{xurl}}{} % add URL line breaks if available
\urlstyle{same} % disable monospaced font for URLs
\hypersetup{
  pdftitle={Copywriting 360°},
  pdfauthor={Karol Leszczyński},
  pdflang={pl},
  hidelinks,
  pdfcreator={LaTeX via pandoc}}


\title{Copywriting 360°}
\usepackage{etoolbox}
\makeatletter
\providecommand{\subtitle}[1]{% add subtitle to \maketitle
  \apptocmd{\@title}{\par {\large #1 \par}}{}{}
}
\makeatother
\subtitle{Od Psychologii do Konwersji}
\author{Karol Leszczyński}
\date{2026-01-01}

\begin{document}
\frontmatter
\maketitle

% ============================================================
% TITLE PAGE
% ============================================================
\thispagestyle{empty}
\begin{center}
\vspace*{4cm}

{\fontsize{48}{52}\selectfont\bfseries\color{primarydark}Copywriting 360°}

\vspace{0.5cm}

{\LARGE\color{textmuted}Od Psychologii do Konwersji}

\vspace{3cm}

{\Large Karol Leszczyński}

\vspace{4cm}

\begin{minipage}{0.7\textwidth}
\centering
{\large\color{primaryblue}Kompletny przewodnik po sztuce copywritingu}

\vspace{1cm}

\color{textmuted}
Psychologia perswazji \textbullet\ Formuły i techniki\\[8pt]
Landing pages \textbullet\ Email marketing \textbullet\ Social media\\[8pt]
AI w copywritingu \textbullet\ Budowanie biznesu
\end{minipage}

\vfill

{\color{textmuted}Wydanie I \textbullet\ 2026}

\vspace{0.5cm}

{\color{primaryblue}\rule{0.3\textwidth}{2pt}}

\end{center}
\cleardoublepage

% ============================================================
% COPYRIGHT PAGE
% ============================================================
\thispagestyle{empty}
\vspace*{\fill}
\begin{center}
\small\color{textmuted}

\textbf{Copywriting 360°}\\
Od Psychologii do Konwersji

\vspace{1.5cm}

Copyright © 2026 Karol Leszczyński\\
Wszelkie prawa zastrzeżone.

\vspace{1.5cm}

Wydanie I

\vspace{2cm}

{\color{primaryblue}\rule{0.15\textwidth}{0.5pt}}

\vspace{1.5cm}

kontakt@kurscopywritingu.pl\\
www.kurscopywritingu.pl

\end{center}
\vspace*{\fill}
\cleardoublepage

% ============================================================
% DEDICATION
% ============================================================
\thispagestyle{empty}
\vspace*{0.3\textheight}
\begin{center}
\large\itshape\color{textmuted}
Dla wszystkich, którzy wierzą,\\
że słowa mogą zmieniać świat.
\end{center}
\vspace*{\fill}
\cleardoublepage

\renewcommand*\contentsname{Spis treści}
{
\setcounter{tocdepth}{2}
\tableofcontents
}

\setstretch{1.15}
\mainmatter
\bookmarksetup{startatroot}

\chapter{Copywriting 360°}\label{copywriting-360}

Od Psychologii do Konwersji

\hfill\break

\bookmarksetup{startatroot}

\chapter*{Przedmowa}\label{przedmowa}
\addcontentsline{toc}{chapter}{Przedmowa}

\markboth{Przedmowa}{Przedmowa}

Trzymasz w rękach efekt ponad 15 lat doświadczenia w marketingu
cyfrowym, copywritingu i budowaniu biznesów online. To nie jest kolejny
zbiór ``sztuczek'' i ``hacków'' --- to kompleksowy system myślenia o
komunikacji perswazyjnej.

\section{Dla kogo jest ta
książka?}\label{dla-kogo-jest-ta-ksiux105ux17cka}

Ta książka jest dla Ciebie, jeśli:

\begin{itemize}
\tightlist
\item
  Prowadzisz biznes i chcesz pisać teksty, które sprzedają
\item
  Jesteś marketerem, który chce podnieść skuteczność kampanii
\item
  Chcesz zostać copywriterem i zarabiać na pisaniu
\item
  Tworzysz content i chcesz, żeby prowadził do działania
\item
  Interesujesz się psychologią perswazji i jej zastosowaniami
\end{itemize}

\section{Jak korzystać z tej
książki?}\label{jak-korzystaux107-z-tej-ksiux105ux17cki}

\textbf{Dla początkujących:} Przerabiaj rozdziały po kolei. Każdy buduje
na poprzednim.

\textbf{Dla średnio zaawansowanych:} Możesz przeskakiwać do
interesujących Cię tematów, ale wracaj do fundamentów, gdy poczujesz
braki.

\textbf{Dla zaawansowanych:} Użyj tej książki jako reference --- wracaj
do konkretnych technik i formuł przy projektach.

\section{Oznaczenia w książce}\label{oznaczenia-w-ksiux105ux17cce}

W całej książce znajdziesz specjalne bloki, które wyróżniają różne typy
treści:

\begin{wskazowka}
Wskazówki to praktyczne porady, które możesz zastosować natychmiast.
\end{wskazowka}

\begin{uwaga}
Uwagi zwracają uwagę na częste błędy lub pułapki.
\end{uwaga}

\begin{wazne}
Bloki "Ważne" podkreślają kluczowe informacje, których nie możesz pominąć.
\end{wazne}

\begin{pojecie}{Nazwa pojęcia}
Definicje kluczowych pojęć, które musisz znać.
\end{pojecie}

\begin{formula}{Nazwa formuły}
Gotowe do użycia formuły i szablony copywriterskie.
\end{formula}

\begin{cwiczenie}{Nazwa ćwiczenia}
Praktyczne zadania do wykonania. Nie pomijaj ich — 70\% nauki to praktyka!
\end{cwiczenie}

\section*{Podziękowania}\label{podziux119kowania}
\addcontentsline{toc}{section}{Podziękowania}

\markright{Podziękowania}

Dziękuję wszystkim klientom, którzy przez lata pozwalali mi testować,
błądzić i doskonalić techniki opisane w tej książce. Dziękuję też Tobie,
Czytelniku --- bez Twojego zaufania ta praca nie miałaby sensu.

Do dzieła!

\emph{Karol Leszczyński}\\
\emph{Luty 2026}

\part{CZĘŚĆ I: FUNDAMENTY}

\chapter{Czym naprawdę jest
copywriting}\label{czym-naprawdux119-jest-copywriting}

I dlaczego zmieni Twoje podejście do biznesu

\hfill\break

\chapteropener{W świecie przesyconym treścią, słowa są Twoją supermocą.\\Naucz się ich używać.}

\textbf{Każdego dnia jesteś bombardowany tysiącami komunikatów
marketingowych.} Reklamy w social mediach, emaile, billboardy, strony
internetowe, opakowania produktów --- wszędzie ktoś próbuje zwrócić
Twoją uwagę i przekonać Cię do działania. Większość tych komunikatów
przepływa przez Twoją świadomość bez śladu. Ale niektóre\ldots{}
niektóre zatrzymują Cię w miejscu. Zmuszają do przeczytania. I
działania.

To nie magia. To \textbf{copywriting}.

\begin{cytat}
Nie sprzedajesz produktu. Sprzedajesz lepszą wersję klienta, którą może dzięki Twojemu produktowi stać.

\hfill--- Donald Miller, autor \textit{Building a StoryBrand}
\end{cytat}

\section{Definicja, która zmienia
wszystko}\label{definicja-ktuxf3ra-zmienia-wszystko}

Zanim zagłębimy się w techniki, musimy ustalić, czym właściwie jest
copywriting. I tutaj pierwsza niespodzianka --- definicja, którą
znajdziesz w większości miejsc, jest niekompletna.

\begin{pojecie}{Copywriting}
\textbf{Copywriting} to sztuka i nauka pisania tekstów, które skłaniają odbiorcę do podjęcia określonego działania --- przy jednoczesnym budowaniu relacji i zaufania.

\medskip
Kluczowe elementy:
\begin{itemize}
  \item \textbf{Sztuka} --- kreatywność, storytelling, emocje
  \item \textbf{Nauka} --- psychologia, testowanie, dane  
  \item \textbf{Działanie} --- konkretny, mierzalny cel
  \item \textbf{Relacja} --- długoterminowe zaufanie, nie manipulacja
\end{itemize}
\end{pojecie}

Zwróć uwagę na ostatni punkt. Wielu copywriterów popełnia błąd,
skupiając się wyłącznie na krótkoterminowej konwersji. Tymczasem
prawdziwy copywriting buduje marki, które przetrwają dekady.

\decoline

\section{Copywriting a inne formy
pisania}\label{copywriting-a-inne-formy-pisania}

Często słyszę pytanie: ``Czym różni się copywriting od content
writingu?'' To fundamentalne rozróżnienie, które musisz zrozumieć.

\begin{longtable}[]{@{}
  >{\raggedright\arraybackslash}p{(\linewidth - 6\tabcolsep) * \real{0.1600}}
  >{\raggedright\arraybackslash}p{(\linewidth - 6\tabcolsep) * \real{0.2600}}
  >{\raggedright\arraybackslash}p{(\linewidth - 6\tabcolsep) * \real{0.3400}}
  >{\raggedright\arraybackslash}p{(\linewidth - 6\tabcolsep) * \real{0.2400}}@{}}
\toprule\noalign{}
\begin{minipage}[b]{\linewidth}\raggedright
Aspekt
\end{minipage} & \begin{minipage}[b]{\linewidth}\raggedright
Copywriting
\end{minipage} & \begin{minipage}[b]{\linewidth}\raggedright
Content Writing
\end{minipage} & \begin{minipage}[b]{\linewidth}\raggedright
UX Writing
\end{minipage} \\
\midrule\noalign{}
\endhead
\bottomrule\noalign{}
\endlastfoot
\textbf{Cel} & Konwersja & Edukacja & Użyteczność \\
\textbf{Horyzont} & Natychmiastowy & Długoterminowy & Momentalny \\
\textbf{Ton} & Perswazyjny & Informacyjny & Neutralny \\
\textbf{Długość} & 3 słowa -- 20 stron & 500--5000+ słów & 1--50 słów \\
\textbf{Przykłady} & Reklamy, landing pages & Blog, poradniki &
Przyciski, komunikaty \\
\end{longtable}

\begin{wskazowka}
Granice między tymi formami zacierają się. Najlepsi specjaliści potrafią płynnie przechodzić między nimi. Artykuł blogowy może mieć elementy copywritingu (CTA), a landing page może edukować. Traktuj te kategorie jako spektrum, nie sztywne podziały.
\end{wskazowka}

\section{Krótka historia
copywritingu}\label{kruxf3tka-historia-copywritingu}

\subsection{Era pionierów (1900--1950)}\label{era-pionieruxf3w-19001950}

Claude Hopkins, David Ogilvy, John Caples --- ci pionierzy ustanowili
fundamenty, które używamy do dziś. Co ciekawe, większość ich zasad
powstała zanim istniał internet, a mimo to doskonale sprawdzają się w
erze cyfrowej.

\begin{cytat}
Reklama to sprzedaż w druku. Traktuj ją jak handlowca, który odwiedza tysiące klientów jednocześnie.

\hfill--- Claude Hopkins, \textit{Scientific Advertising} (1923)
\end{cytat}

\subsection{Era direct response
(1950--1990)}\label{era-direct-response-19501990}

To złoty wiek direct mail --- długich listów sprzedażowych, które
generowały miliony dolarów. Gary Halbert, Dan Kennedy, Eugene Schwartz
udoskonalili techniki perswazji do poziomu niemal naukowej precyzji.

\stathighlight{30:1}{Legendarny list Halberta przyniósł 30 mln USD przy inwestycji 1 mln USD}

\subsection{Era digitalna
(1990--obecnie)}\label{era-digitalna-1990obecnie}

Internet nie zmienił fundamentów copywritingu --- zmienił kanały
dystrybucji i możliwości testowania. Dziś możesz w ciągu godziny
przetestować 10 wersji nagłówka i dowiedzieć się, która działa lepiej.
To rewolucja w szybkości iteracji.

\begin{wazne}
Nie daj się zwieść: mimo AI i automatyzacji, podstawy copywritingu --- psychologia, struktura, jasność przekazu --- pozostają niezmienne. Narzędzia się zmieniają, ludzie nie.
\end{wazne}

\decoline

\section{Anatomia konwersji}\label{anatomia-konwersji}

Zanim napiszesz pierwsze słowo, musisz zrozumieć, co właściwie próbujesz
osiągnąć. W copywritingu wszystko sprowadza się do \textbf{konwersji}.

\begin{pojecie}{Konwersja}
\textbf{Konwersja} to moment, w którym odbiorca Twojego komunikatu wykonuje pożądane działanie --- kliknięcie, zapis, zakup, telefon, pobranie.

\medskip
\textbf{Ważne:} Konwersja to NIE zawsze sprzedaż. Zależy od etapu ścieżki klienta i celu komunikatu.
\end{pojecie}

\subsection{Mikro vs.~makro konwersje}\label{mikro-vs.-makro-konwersje}

\textbf{Makro konwersja} to główny cel biznesowy --- np. zakup produktu
za 997 zł.

\textbf{Mikro konwersje} to mniejsze kroki prowadzące do makro
konwersji:

\begin{itemize}
\tightlist
\item
  Kliknięcie w reklamę
\item
  Zapis na newsletter\\
\item
  Pobranie lead magneta
\item
  Otwarcie emaila
\item
  Dodanie do koszyka
\end{itemize}

Każda mikro konwersja wymaga osobnego ``copy'' --- emaila, przycisku,
strony. Dobry copywriter myśli o całej ścieżce, nie tylko o końcowym
kroku.

\section{Co czyni copy skutecznym?}\label{co-czyni-copy-skutecznym}

Po przeanalizowaniu setek kampanii, mogę wyodrębnić \textbf{5 cech},
które wyróżniają copy generujące wyniki:

\begin{enumerate}
\def\labelenumi{\arabic{enumi}.}
\tightlist
\item
  \textbf{Jasność ponad wszystko} --- Odbiorca musi zrozumieć przekaz w
  3 sekundy
\item
  \textbf{Korzyść, nie cecha} --- „Oszczędzisz 10 godzin tygodniowo'',
  nie „Automatyzacja procesów''
\item
  \textbf{Emocja + logika} --- Emocja przekonuje, logika usprawiedliwia
  decyzję
\item
  \textbf{Konkretność} --- „37\% więcej konwersji'', nie „więcej
  konwersji''
\item
  \textbf{Wezwanie do działania} --- Jasne, jednoznaczne, pilne
\end{enumerate}

\begin{beforeafter}
{Nasza innowacyjna platforma oferuje kompleksowe rozwiązania w zakresie automatyzacji procesów biznesowych, umożliwiając przedsiębiorstwom osiągnięcie przewagi konkurencyjnej na dynamicznie zmieniającym się rynku.}
{Zaoszczędź 10 godzin tygodniowo dzięki automatyzacji faktur. Dołącz do 2,347 firm, które już to zrobiły. Sprawdź demo →}
\end{beforeafter}

\section{Mity o copywritingu}\label{mity-o-copywritingu}

Zanim przejdziemy dalej, rozprawmy się z kilkoma szkodliwymi mitami:

\begin{uwaga}
\textbf{MIT 1: "Copywriting to manipulacja"}

Manipulacja to świadome wprowadzanie w błąd dla własnej korzyści. Etyczny copywriting to jasne komunikowanie wartości produktu osobom, które naprawdę mogą z niego skorzystać. Różnica jest fundamentalna.

\medskip
\textbf{MIT 2: "Trzeba być urodzonym pisarzem"}

Copywriting to umiejętność --- można się jej nauczyć. Najlepsi copywriterzy to często byli inżynierowie, sprzedawcy, psychologowie. Liczy się zrozumienie ludzi, nie talent literacki.

\medskip
\textbf{MIT 3: "AI zastąpi copywriterów"}

AI to narzędzie, jak kalkulator dla matematyka. Może przyspieszyć pracę, ale strategia, empatia i zrozumienie kontekstu pozostają domeną człowieka. Więcej o tym w Module 19.
\end{uwaga}

\decoline

\section{Struktura tego kursu}\label{struktura-tego-kursu}

Ten kurs jest podzielony na \textbf{6 części}, które prowadzą Cię od
fundamentów do zaawansowanych technik:

\begin{longtable}[]{@{}lll@{}}
\toprule\noalign{}
Część & Nazwa & Zakres \\
\midrule\noalign{}
\endhead
\bottomrule\noalign{}
\endlastfoot
I & FUNDAMENTY & Psychologia, research, poziomy świadomości \\
II & TECHNIKI & Nagłówki, formuły, storytelling, obiekcje \\
III & FORMATY & Landing pages, email, social, e-commerce, B2B, UX \\
IV & AI I NARZĘDZIA & Prompt engineering, toolkit, testowanie \\
V & BIZNES & Portfolio, wycena, pozyskiwanie klientów \\
VI & PROJEKT & Capstone z feedbackiem \\
\end{longtable}

\begin{wskazowka}
Nie musisz przerabiać kursu linearnie. Jeśli jesteś doświadczonym marketerem, możesz zacząć od Części III (formaty). Jeśli chcesz szybko pozyskać klientów --- skocz do Części V. Wracaj do wcześniejszych modułów, gdy poczujesz, że brakuje Ci fundamentów.
\end{wskazowka}

\section{Jak maksymalnie wykorzystać ten
kurs}\label{jak-maksymalnie-wykorzystaux107-ten-kurs}

\begin{checklist}
  \item \textbf{Rób ćwiczenia} --- 70\% nauki to praktyka, nie teoria
  \item \textbf{Zbieraj swipe file} --- Zapisuj przykłady dobrego copy
  \item \textbf{Analizuj codziennie} --- Patrz na reklamy przez pryzmat technik z kursu
  \item \textbf{Pisz codziennie} --- Nawet 15 minut dziennie robi różnicę
  \item \textbf{Szukaj feedbacku} --- Pokaż swoje copy innym
\end{checklist}

\decoline

\begin{cwiczenie}{Audyt własnego copy}
Zanim przejdziesz dalej, wykonaj pierwszą analizę:

\begin{enumerate}
  \item Znajdź jedno \textbf{własne} copy (email, post, strona) lub copy \textbf{konkurenta}
  \item Przeczytaj je na głos --- gdzie się potykasz? Co brzmi nienaturalnie?
  \item Odpowiedz na pytania:
  \begin{itemize}
    \item Czy w 3 sekundy wiem, o co chodzi?
    \item Czy jest jasna korzyść dla mnie?
    \item Czy wiem, co mam zrobić dalej?
  \end{itemize}
  \item Zapisz 3 rzeczy do poprawy
\end{enumerate}

\textbf{Czas:} 15--20 minut
\end{cwiczenie}

\takeaway{Copywriting to umiejętność łącząca psychologię, kreatywność i strategię. Nie jest manipulacją --- jest jasną komunikacją wartości. I jak każda umiejętność, można się go nauczyć poprzez systematyczną praktykę i analizę.}

W następnym rozdziale zanurzymy się w \textbf{psychologię perswazji} ---
naukowy fundament, który sprawi, że Twoje copy będzie działać nie przez
przypadek, ale przez głębokie zrozumienie, jak ludzki mózg podejmuje
decyzje.

\decoline


\backmatter


\end{document}
